\section{SubProgram}
\textbf{pass by value:}The value of the actual parameter is used to initialize the corresponding formal parameter. advantage: fast, disadvantage: additional storage
\textbf{Example} Java;\\
\textbf{pass by reference} pass an access path. advantage no copying or duplicate storage. disadvantage: slower access, potential unwanted side effects.
\textbf{Example} C++\\
\textbf{pass by value-result:} combination of pass by value and pass by result. disadvantage is copy storage, 
\textbf{Example} Fortrans\\
\textbf{pass by name:} by textual substitution. Formals are bound to an access method at the time of the call, but actual binding to a value or address takes place at the time of a reference or assignment. Allows flexibility in late binding. Implementation requires that the referencing environment of the caller is passed with the parameter, so the actual parameter address can be calculated. Disadvantages: Very tricky hard to read and understand. Essentially, the body of a function is interpreted at call time after textually substituting the actual parameters into the function body. In this sense the evaluation method is similar to that of C preprocessor macros.

By substituting the actual parameters into the function body, the function body can both read and write the given parameters. In this sense the evaluation method is similar to pass-by-reference. The difference is that since with pass-by-name the parameter is evaluated inside the function, a parameter such as a[i] depends on the current value of i inside the function, rather than referring to the value at a[i] before the function was called.


\textbf{Example:}Algol 60 \\

\texbf{Example home work 4 and problem 9}\\

\textbf{SubProgram as parameter}
\textbf{Why do we do that, what is the advantage and disadvantage:}\\
advantage: very flexible, like the call back function that make the server works faster.
disadvantage: hard to understand, bad readability. 
function sub1() {
var x;
function sub2() {
alert(x); // Creates a dialog box with the value of x
};
function sub3() {
var x;
x = 3;
sub4(sub2);
};
function sub4(subx) {
var x;
x = 4;
subx();
};
x = 1;
sub3();
};

Consider the execution of sub2 when it is called in sub4. For shallow
binding, the referencing environment of that execution is that of sub4, so the
reference to x in sub2 is bound to the local x in sub4, and the output of the
program is 4. For deep binding, the referencing environment of sub2’s execution
is that of sub1, so the reference to x in sub2 is bound to the local x in
sub1, and the output is 1. For ad hoc binding, the binding is to the local x in
sub3, and the output is 3.


\textbf{Overloaded subprogram}: An overloaded subprogram is one that has the same name as another subprogram in the same referencing environment but with different or unique protocols. Ada, Java, C++, and C# allow users to write multiple versions of subprograms with the same name. \textbf{advantage}: very flexible, you can use a function with just the protocols you need, therefore the programmer don't have to memorize the header and signature of the method.\textbf{A polymorphic subprogram} takes parameters of different types on different activations.\textbf{Subtype polymorphism} means that a variable of type T can access any object of type T or any type derived from T.

\textbf{Generic SubProgram}: Parametric polymorphism is provided by a subprogram that takes
generic parameters that are used in type expressions that describe the types
of the parameters of the subprogram. Different instantiations of such subprograms
can be given different generic parameters, producing subprograms that
take different types of parameters. Parametric definitions of subprograms all
behave the same. Parametrically polymorphic subprograms are often called
generic subprograms.