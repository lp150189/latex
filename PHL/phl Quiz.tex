\documentclass{article}
\usepackage{setspace}
\begin{document}
\author{Huy Le}
\title{ PHL 202 Quiz}
\maketitle
\doublespacing
\indent The first speaker is from MIT. He discusses in his speech about the how every one of us human is also a robot. Robot is defined by him as very similar to animal and us. He suggested that we are similar to robot in the aspect that we can build any human or robot from scratch level. He claimed that it is possible to build a human level being from other material in principle. In other words,  But many people doubt that if we are smart enough to do such a thing. He then went further to prove that human and robot are the same by pointing out that the more we build robots with human characteristics, people start to feel these robot as human. For example, a robot fools human to believe it is human, but that robot will be considered human too if that robot keeps fooling human for 10 years or for a life-time. Finally, He talked about the fear of robot from us. He believe these fears are normal because human afraid of changes. For example, we were afraid of Darwin's revolution theory, when we learned about genome, we learned that we has less genes than a potatoes.\\

\indent The second speaker is Rosalin Picard. She first raised the question that how we can define human. According to her, after many researches, we can't really clearly define what is human. She somewhat agreed with the first speaker by saying that interacting with something for a long time tends to make you think it's human. Then Rosalin Picard raised a very interesting question which is "What are necessary and sufficient conditions for a robot to be called human". And her answer there is no conditions for robot to be called human. She believed that robots can never called human because we can't build emotions, experience feelings, free will to them like us. She ended her speech by including the chapter of love from bible and saying that we are fully known by god. Someday we might know what is it to be human\\

\indent I believe in the first speaker's opinion since it is the closet to science and make the most sense. Even though the second speaker's point seems to be valid in some ways that we might not be able to create robots with all the human characteristics, I believe that we will be able to do that in future when our science grows big enough. The example from the first speaker when he said that we can build virus completely from the lab means that in the future, we could do the same thing for human. We might not be able to build a complete human robot at the beginning but we will have prototype and one day we will be able to clone ourself with science.\\

\end{document}