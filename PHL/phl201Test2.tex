\documentclass{article}
\usepackage{setspace}
\usepackage[margin=.65in]{geometry}
\begin{document}
\title{PHL 202 Test 2}
\author{Huy Le}
\maketitle
\doublespacing
\textbf{Question 1:} For Descartes, what form of knowledge can be proven as incorrigible through methodical doubt? Why is it incorrigible? How can we use his proof of its incorrigibility as an illustration for the rationalist claim that reason(not the senses ) is sufficient guide to the truth? What is the relevance of such proof in establishing the existence of the mind as something that exists independently of the body? How can the foundations of scientific and mathematical knowledge be considered corrigible through methodical doubt?\\
\indent \textbf{Answer: } The form of knowledge can be proven as incorrigible is the existence of the mind. According to Descartes, as long as I'm thinking, I must exist. Through his method of doubt, we can doubt anything in our worlds like our senses, fundamental Mathematics, Physics, etc but we cannot doubt the existence of the mind because it is the sign of our very existence. We can be deceived by our senses, programmed by evil genius, or even living in a virtual environment buy the moment that we doubt our existence or the moment that we think means that we must exist(or our mind must exist).\\
\indent Descartes proof implies that reason is sufficient guide to the truth or knowledge but not our senses because our senses can be deceived, each of us will look at the same thing differently. Therefore the truth needs to be derived from reasons not from our senses.\\
\indent The body is the external and extended things. It has the existence outside of the brain. Moreover, there are things that outside of our brain and our brain doesn't produce these thing. For example, our knowledge of objects obtained by our senses. Things that we experience must come from something external like our body or from god, but god is not a deceiver.  Therefore, the body must exist independently from the mind. \\
\indent The foundations of scientific and mathematical knowledge be considered corrigible through methodical doubt by using fundamental facts. Scientific and mathematical knowledge are built on fundamental facts that acquired by our senses. Even though our senses can still be doubted but they are at some degree reliable and the fundamental facts we acquire are more solid and harder to doubt.

\indent \textbf{Question 2:} In your own words, briefly state Papineau's casual argument for materialism. What is the abstract claim that he consider to be a necessary presupposition of his argument? Why is it important for him to presuppose such a claim? what is his general response to any objection that denies any of his premises?\\
\indent \textbf{Answer: } Papineau's casual argument for materialism is
\begin{itemize}
\item \textbf{Premise 1:} Conscious mental occurrences have physical effects
\item \textbf{Premise 2:} All physical effects are fully caused by purely physical prior histories
\item \textbf{Premise 3:} The physical effects of conscious causes are not always overdetermined by distinct causes
\item \textbf{Conclusion:} Therefore, the consciousness causes must be identical with physical causes.
\end{itemize}

\indent Premise one means that mental causes have the physical effects. For example, itch, pain, thirst cause us to scratch, say "ouch", or drink. Premise 2 means that each physical effects has physical causes. For example, if you are scratching, then there is some neuron brain event causing you to do that. Premise 3 means that there are no more than one causes for one physical effects. For example, the program isn't running because the computer is broken and the databases is messed up. This can't be true because you only need one cause for the program to not running. With that three premises, Papineau raised a question that why we have to consider the existence of the brain when we can explain every physical effects by look at the physical events happening in our brain(neuron events). This question implies the conclusion of Papineau's casual argument which is every physical effect is caused by physical causes not metal or conscious causes.\\

\indent There is an abstract claim which is all physical effects are reducible into physical cause that he consider to be a necessary presupposition of his argument. It is important for him to presuppose such a claim because that will assume that anything can be reducible into physical even mental occurrences. This presupposition is a theoretical claim that is proven or demonstrated conclusively. Without this presupposition, there would be claims saying that there are some mental occurrences that is outside of the physical world which can disproves his argument.\\ 

\indent Often objections for this argument is possible and people usually give objections by reject one of the three premises. But if you reject any of these premise, you will end up with a weaker argument. If you reject premise one, then pain doesn't cause you to say ouch which is not counterintuitive. Rejecting premise 2 would reject the completeness of physics. In other words, there is something beyond the physical world that we don't know about. Finally, if we reject premise 3, then if I'm scratching, it is caused by 2 causes which is my mind cause me to scratch and the itch cause my to scratch. This might actually be true but then again, you will end up with a weaker argument if you assume that \\

\textbf{Question 3:} Give an account of Jackson's possible objection to Papineau's abstract claim. Based on Jackson's knowledge argument, give an account of his case for his objection. Critics of Jackson's Epiphenomenalism generally point out that there is something counterintuitive about the claim that mental states do not causally interact with physical states. Give an account of Jackson's response to such critics.\\
\indent \textbf{Answer: }\\
\indent Jackson objection to Papineau's abstract claim by saying that there is at least one kind of thing that we can't be reducible to physical namely Qualia. %need hearing 
Qualia is the subjective quality of experience. For example, the pain that you experience, that there is no way that people that can experience the pain that you experience. Another example would be that when you are actually scratching, you can describe to me all you want how itchy you are but only you know how itchy it is. He used theoretical experiment to demonstrate Qualia and support his objection. Example: There is a guys named Fred and he can see the color that is only accessible to him. There are a bunch of tomatoes with same color and similar sizes. Fred entered the room, and he said that we have two kind of tomatoes which are green and red. Then we asked him to separate two types of tomatoes. He will separate these two tomatoes. We will cover his eyes and mark the green tomatoes. Then we ask him to separate these tomatoes again. Fred will separate the same way he did the first time. We would ask him to separate the tomatoes with these same steps. Fred keep separating these tomatoes the same way he did. Therefore, we can conclude that there is a color that is only accessible to Fred. Moreover, no matter what we try to analyze Fred brain we could never know the color that Fred see. According to Jackson, you can study everything about one person's brain but there will be always something left out. One of the critics to Jackson's Epiphenomenalism would be the claim that "We know about other's minds by knowing about other's behaviors". For example, when you are scratching, you must be itching. You think automatically of something cause you to do some behavior. But Jackson said that you don't have to explain it that way. He believe that when you say ouch or experience the Qualia of pain, there are some brain event that causing you to say ouch and the same brain event that cause you the pain.


\textbf{Question 6:} What are the three conditions that need to be met in order for a claim to as as knowledge? Describe the nature of Gettier's criticism of this understanding of knowledge. How does the counter-example from Gettier that we discussed in class disprove such an understanding of knowledge?\\
\indent \textbf{Answer: }\\
\indent There are three conditions for a claim or proposition to become knowledge. First, you have to believe in your claim or your proposition. Secondly, the claim that you propose must be true. Lastly, that claim is justified. Even though the three conditions sound separating from each other, they are very highly related. The first condition implement the second condition. If I believe in a claim, then that claim is also true. In other words, claims entails truth. For example; I believe the sky is blue then the sky is blue is also a truth. The third condition which is the claim need to be justified means that knowledge includes more than having a true belief. It must include adequate justification for belief. For example: My name is Huy and I say "Huy's GPA is 2.0", and my friend is saying the same claim. My claims is considered knowledge because it is justified. My friend's claim is not knowledge because he claim the true belief accidentally without any solid justification. In the end, these three conditions are necessary and sufficient for a claim to become knowledge.\\
\indent Gettier criticizes this understanding of knowledge by saying that the three conditions for a claim to become knowledge are necessary but not sufficient. He used the counter-example to disprove this understanding of knowledge. His method was finding an example that qualifies all three conditions and then that example is still not true. Therefore, using the three conditions evaluate a claim to be knowledge is wrong. In Gettier's example, suppose that there are two people who are Smith and Jones applying for the same job. Smith has a law degree from Harvard, economy degree from MIT, and also seven years in the field. Jones has law degree from Lavern and three years of experience. You will then be justified believing Smith will get the job, and let's say that Smith has 10 coins in his pocket. From that, Jones can claim that the one who get the job has 10 coins in his pocket because he concludes based on the claim before that. But in the end, Jones get the job and Jones didn't know that he has 10 coins in his pocket. Interestingly, the first claim about Smith get the job was wrong but the second claim that person who gets the job has 10 coins in his pocket remains true. Also, all three conditions are satisfied for the second claim which is the person who gets the job has 10 coins in his pocket. Firstly, Jones believe in the second claim. Secondly, the second claim is actually true. Thirdly, the second claim is actually justified. But can we conclude that Jones knows that the person who gets the job has 10 coins in his pocket? In other word, can the second claim become knowledge because the three conditions are satisfied? The answer is no because that claim is only accidentally true. Therefore, he disproved this way of understanding of knowledge.\\
\end{document}